\documentclass{article}[11pt]

\usepackage{amsmath}
\usepackage{amsfonts}
\usepackage{proof}
\usepackage{stmaryrd}

\begin{document}
\section{A Confession}
I suck at \LaTeX.

\section{Syntax}
$$
\begin{aligned}
x,y,z,w \in \mathit{Variable}\\
t,t',t'' \in \mathit{Type} &::= ()~ |~ t + t'~ |~ t \times t'~ |~ t \rightarrow t'~ |~ M(t)\\
d \in \mathit{DataType} &::= \text{a restriction of}~ \mathit{Type}~ \text{to ``data-only'' types}\\
M \in \mathit{Monad} &::= \mathbb{I}~ |~ T M\\
T \in \mathit{MonadTrans} &::= \mathbb{R}_{t:t'}~ |~ \mathbb{S}_t~ |~ \mathbb{E}_t~ |~ \mathbb{M}_{d:t}\\
e,e',e'' \in \mathit{Exp}&::= x~ |~ \lambda{}x_t\rightarrow{}e~ |~ e~ e'&&\text{(lambda calculus)}\\
  &~~~~|~ \textbf{nil}&&\text{(unit)}\\
  &~~~~|~ \textbf{inl}_{t,t'}~ |~ \textbf{inr}_{t,t'}~ |~ \textbf{case}~ e~ \textbf{of}~ \textbf{inl}~ x \rightarrow e'~ ; \textbf{inr}~ y \rightarrow e''~ \textbf{end}&&\text{(coproduct)}\\
  &~~~~|~ \textbf{mkpair}_{t,t'}~ |~ \mathbf{fst}_{t,t'}~ |~ \mathbf{snd}_{t,t'}&&\text{(product)}\\
  &~~~~|~ \textbf{return}_{t,M}~ |~ \textbf{bind}_{t,t',M}~ |~ \textbf{lift}_{t,M,T}&&\text{(monad operators)}\\
  &~~~~|~ \textbf{signal}_{t,t',M}&&\text{(reactive prims.)}\\
  &~~~~|~ \textbf{get}_{t,M}~ |~ \textbf{put}_{t,M}&&\text{(state prims.)}\\
  &~~~~|~ \textbf{throw}_{t,t',M}~ |~ \textbf{catch}_{t,t',M}&&\text{(error prims.)}\\
  &~~~~|~ \textbf{rdmem}_{t,t',M}~ |~ \textbf{wrmem}_{t,t',M}&&\text{(memory prims.)}
\end{aligned}
$$

\subsection{Notes}
Note: $\mathbb{R}$ is {\it ReactT}, $\mathbb{S}$ is {\it StateT}, $\mathbb{E}$ is {\it ErrorT}, and $\mathbb{M}$ is {\it MemT} (for addressable memory, which may or may not actually be synthesized as RAM).

In reality we're going to support non-recursive, first-order data types with named constructors, but for the abstract presentation, sums/products/unit will do (e.g. $Bit$ can be encoded as $() + ()$).

For the moment I am trying to avoid polymorphism entirely, meaning any expression constructor that on its own might have polymorphic type has a type subscript. The ``real'' language could have Hindley-Milner style polymorphism and of course on paper it's fine to leave off subscripts when they can be inferred from context.

The restriction of the domain type in $\mathbb{M}$ is not accidental. It's there because even in a higher-order setting we need to be able to test for equality.

It may not be necessary to make {\it MemT} a special primitive; memory extraction could probably be performed from {\it StateT}.

\section{Type System}
\infer[\textsc{T-Lam}]{\Gamma\vdash{}\lambda{}x_t\rightarrow{}e : t \rightarrow t'}{\Gamma,x : t\vdash{}e : t'}
\vspace{1em}
\infer[\textsc{T-Var}]{\Gamma,x : t\vdash{}x : t}{}
\vspace{1em}
\infer[\textsc{T-App}]{\Gamma\vdash{}e~ e' : t}{\Gamma\vdash{}e : t' \rightarrow t~~~\Gamma\vdash{}e' : t'}
\vspace{1em}
\infer[\textsc{T-Nil}]{\Gamma\vdash{}\textbf{nil} : ()}{}
\vspace{1em}
\infer[\textsc{T-InL}]{\Gamma\vdash{}\textbf{inl}_{t,t'} : t \rightarrow t + t'}{}
\vspace{1em}
\infer[\textsc{T-InR}]{\Gamma\vdash{}\textbf{inr}_{t,t'} : t' \rightarrow t + t'}{}
\vspace{1em}
\infer[\textsc{T-Case}]{\Gamma\vdash{}\textbf{case}~ e~ \textbf{of}~ \textbf{inl}~ x \rightarrow e'~ ; \textbf{inr}~ y \rightarrow e''~ \textbf{end} : t}{\Gamma\vdash{}e : t' + t''~~~~\Gamma,x : t'\vdash{}e' : t~~~~\Gamma,y : t''\vdash{}e'' : t}
\vspace{1em}
\infer[\textsc{T-MkPair}]{\Gamma\vdash{}\textbf{mkpair}_{t,t'} : t \rightarrow t' \rightarrow t \times t'}{}
\vspace{1em}
\infer[\textsc{T-Pi1}]{\Gamma\vdash{}\mathbf{fst}_{t,t'} : t \times t' \rightarrow t}{}
\vspace{1em}
\infer[\textsc{T-Pi2}]{\Gamma\vdash{}\mathbf{snd}_{t,t'} : t \times t' \rightarrow t'}{}
\vspace{1em}
\infer[\textsc{T-Return}]{\Gamma\vdash{}\textbf{return}_{t,M} : t \rightarrow M(t)}{}
\vspace{1em}
\infer[\textsc{T-Bind}]{\Gamma\vdash{}\textbf{bind}_{t,t',M} : M(t) \rightarrow (t \rightarrow M(t')) \rightarrow M(t')}{}
\vspace{1em}
\infer[\textsc{T-Lift}]{\Gamma\vdash{}\textbf{lift}_{t,M,T} : M(t) \rightarrow TM(t)}{}
\vspace{1em}
\infer[\textsc{T-Signal}]{\Gamma\vdash{}\textbf{signal}_{t,t',M} : t' \rightarrow \mathbb{R}_{t\rightarrow{}t'}M(t)}{}
\vspace{1em}
\infer[\textsc{T-Get}]{\Gamma\vdash{}\textbf{get}_{t,M} : \mathbb{S}_tM(t)}{}
\vspace{1em}
\infer[\textsc{T-Put}]{\Gamma\vdash{}\textbf{put}_{t,M} : t \rightarrow \mathbb{S}_tM\big(()\big)}{}
\vspace{1em}
\infer[\textsc{T-Throw}]{\Gamma\vdash{}\textbf{throw}_{t,t',M} : t' \rightarrow \mathbb{E}_{t'}M(t)}{}
\vspace{1em}
\infer[\textsc{T-Catch}]{\Gamma\vdash{}\textbf{catch}_{t,t',M} : \mathbb{E}_{t'}M(t) \rightarrow (t' \rightarrow \mathbb{E}_{t'}M(t)) \rightarrow \mathbb{E}_{t'}M(t)}{}
\vspace{1em}
\infer[\textsc{T-RdMem}]{\Gamma\vdash{}\textbf{rdmem}_{t,t',M} : t \rightarrow \mathbb{M}_{t\rightarrow{}t'}M(t')}{}
\vspace{1em}
\infer[\textsc{T-WrMem}]{\Gamma\vdash{}\textbf{wrmem}_{t,t',M} : t \rightarrow t' \rightarrow \mathbb{M}_{t\rightarrow{}t'}M\big(()\big)}{}

\section{Denotational Semantics}

\subsection{Of Types}
\begin{eqnarray*}
\llbracket{}()\rrbracket &=& \{()\}\\
\llbracket{}t+t'\rrbracket &=& \llbracket{}t\rrbracket + \llbracket{}t'\rrbracket\\
\llbracket{}t\times{}t'\rrbracket &=& \llbracket{}t\rrbracket \times \llbracket{}t'\rrbracket\\
\llbracket{}t\rightarrow{}t'\rrbracket &=& \llbracket{}t'\rrbracket^{\llbracket{}t\rrbracket}\\
\llbracket{}M(t)\rrbracket &=& \mathcal{M}\llbracket{}M\rrbracket\llbracket{}t\rrbracket
\end{eqnarray*}

\subsection{Of Monads}

\subsubsection{Type Constructors}
\begin{eqnarray*}
\mathcal{M}\llbracket{}\mathbb{I}\rrbracket &=& \lambda t . t\\
\mathcal{M}\llbracket{}\mathbb{R}_{t:t'}M\rrbracket &=& \lambda t'' . \nu{}X . \mathcal{M}\llbracket{}M\rrbracket (t'' + (\llbracket{}t'\rrbracket \times (\llbracket{}t\rrbracket \rightarrow X)))\\
\mathcal{M}\llbracket{}\mathbb{S}_t M\rrbracket &=& \lambda t' . \llbracket{}t\rrbracket \rightarrow \mathcal{M}\llbracket{}M\rrbracket (t' \times \llbracket{}t\rrbracket)\\
\mathcal{M}\llbracket{}\mathbb{E}_t M\rrbracket &=& \lambda{} t' . \mathcal{M}\llbracket{}M\rrbracket (\llbracket{}t\rrbracket + t')\\
\mathcal{M}\llbracket{}\mathbb{M}_{t:t'} M\rrbracket &=& \mathcal{M}\llbracket{}\mathbb{S}_{t\rightarrow{}t'} M\rrbracket\\
\end{eqnarray*}

\subsubsection{Unit Operators}
\begin{eqnarray*}
\mathcal{U}\llbracket{}\mathbb{I}\rrbracket &=& \lambda x . x\\
\mathcal{U}\llbracket{}\mathbb{R}_{t:t'}M\rrbracket &=& \mathcal{U}\llbracket{}M\rrbracket \circ \mathbf{inl}\\
\mathcal{U}\llbracket{}\mathbb{S}_t M\rrbracket &=& \lambda x . \lambda \sigma . \mathcal{U}\llbracket{}M\rrbracket \langle x,\sigma \rangle\\
\mathcal{U}\llbracket{}\mathbb{E}_t M\rrbracket &=& \mathcal{U}\llbracket{}M\rrbracket \circ \mathbf{inr}\\
\mathcal{U}\llbracket{}\mathbb{M}_{t:t'} M\rrbracket &=& \mathcal{U}\llbracket{}\mathbb{S}_{t\rightarrow{}t'} M\rrbracket\\
\end{eqnarray*}

\subsubsection{Bind Operators}
\begin{eqnarray*}
\mathcal{B}\llbracket{}\mathbb{I}\rrbracket &=& \lambda x . \lambda f . f x\\
\mathcal{B}\llbracket{}\mathbb{R}_{t:t'}M\rrbracket &=& \mathbf{fix}~ F . \lambda \varphi . \lambda f . \mathcal{B}\llbracket{}M\rrbracket \varphi \left(\lambda r . \begin{cases}f x&\text{if } r = \mathbf{inl}~ x\\\mathcal{U}\llbracket{}M\rrbracket (\mathbf{inr} \langle o,\lambda i . F~ (\kappa~ i)~ f\rangle)&\text{if } r = \mathbf{inr} \langle{}o,\kappa\rangle\end{cases}\right)\\
\mathcal{B}\llbracket{}\mathbb{S}_t M\rrbracket &=& \lambda \varphi . \lambda f . \lambda s . \mathcal{B}\llbracket{}M\rrbracket (\varphi s) (\lambda r . f (\pi_2r) (\pi_1 r))\\
\mathcal{B}\llbracket{}\mathbb{E}_t M\rrbracket &=& \lambda \varphi . \lambda f . \mathcal{B}\llbracket{}M\rrbracket \varphi \left(\lambda r . \begin{cases}\mathcal{U}\llbracket{}M\rrbracket{}r&\text{if } r = \mathbf{inl}~ e\\f x&\text{if } r = \mathbf{inr}~ x\end{cases}\right)\\
\mathcal{B}\llbracket{}\mathbb{M}_{t:t'} M\rrbracket &=& \mathcal{B}\llbracket{}\mathbb{S}_{t\rightarrow{}t'} M\rrbracket\\
\end{eqnarray*}

\subsubsection{Lift Operators}
\begin{eqnarray*}
\mathcal{L}\llbracket{}M,\mathbb{R}_{t:t'}\rrbracket &=& \lambda \varphi . \mathcal{B}\llbracket{}M\rrbracket \varphi (\mathcal{U}\llbracket{}M\rrbracket \circ \mathbf{inl})\\
\mathcal{L}\llbracket{}M,\mathbb{S}_t\rrbracket &=& \lambda \varphi . \lambda \sigma . \mathcal{B}\llbracket{}M\rrbracket \varphi (\lambda x . \mathcal{U}\llbracket{}M\rrbracket\langle{}x,\sigma\rangle)\\
\mathcal{L}\llbracket{}M,\mathbb{E}_t\rrbracket &=& \lambda \varphi . \mathcal{B}\llbracket{}M\rrbracket \varphi (\mathcal{U}\llbracket{}M\rrbracket \circ \mathbf{inr})\\
\mathcal{L}\llbracket{}M,\mathbb{M}_{t:t'}\rrbracket &=& \mathcal{L}\llbracket{}M,\mathbb{S}_{t\rightarrow{}t'}\rrbracket
\end{eqnarray*}

\subsection{Of Expressions}
For clarity's sake (?), irrelevant type subscripts are dropped.

\begin{eqnarray*}
\llbracket{}x\rrbracket{}\rho &=& \rho x\\
\llbracket{}\lambda x_t \rightarrow e\rrbracket \rho &=& \lambda v \rightarrow \llbracket{}e\rrbracket(\rho[x\mapsto{}v])\\
\llbracket{}e~ e'\rrbracket \rho &=& \llbracket{}e\rrbracket\rho~ (\llbracket{}e'\rrbracket\rho)\\
\llbracket{}\mathbf{nil}\rrbracket \rho &=& ()\\
\llbracket{}\mathbf{inl}\rrbracket \rho &=& \mathbf{inl}\\
\llbracket{}\mathbf{inr}\rrbracket \rho &=& \mathbf{inr}\\
\llbracket{}\mathbf{case}~ e~ \mathbf{of}~ \mathbf{inl}~ x \rightarrow e' ~; \mathbf{inr}~ y \rightarrow e''\rrbracket \rho &=& \begin{cases}
                                                                                                                                 \llbracket e' \rrbracket (\rho[x\mapsto{}v])&\text{if } \llbracket e \rrbracket \rho = \mathbf{inl}~ v\\
                                                                                                                                 \llbracket e'' \rrbracket (\rho[y\mapsto{}v])&\text{if } \llbracket e \rrbracket \rho = \mathbf{inr}~ v\\
                                                                                                                                \end{cases}\\
\llbracket{}\mathbf{mkpair}\rrbracket{}\rho &=& \lambda x . \lambda y . \langle x,y \rangle\\
\llbracket{}\mathbf{fst}\rrbracket\rho &=& \pi_1\\
\llbracket{}\mathbf{snd}\rrbracket\rho &=& \pi_2\\
\llbracket{}\mathbf{return}_M\rrbracket{}\rho &=& \mathcal{U}\llbracket{}M\rrbracket\\
\llbracket{}\mathbf{bind}_M\rrbracket{}\rho &=& \mathcal{B}\llbracket{}M\rrbracket\\
\llbracket{}\mathbf{lift}_{M,T}\rrbracket\rho &=& \mathcal{L}\llbracket{}M,T\rrbracket\\
\llbracket{}\mathbf{signal}_M\rrbracket\rho &=& \lambda o . \mathcal{U}\llbracket{}M\rrbracket(\mathbf{inr}~ \langle o,\mathcal{U}\llbracket{}\mathbb{R}M\rrbracket\rangle)\\
\llbracket{}\mathbf{get}_M\rrbracket\rho &=& \lambda \sigma . \mathcal{U}\llbracket{}M\rrbracket\langle\sigma,\sigma\rangle\\
\llbracket{}\mathbf{put}_M\rrbracket\rho &=& \lambda \sigma' . \lambda \sigma . \mathcal{U}\llbracket{}M\rrbracket\langle(),\sigma'\rangle\\
\llbracket{}\mathbf{throw}_M\rrbracket\rho &=& \mathcal{U}\llbracket{}M\rrbracket \circ \mathbf{inl}\\
\llbracket{}\mathbf{catch}_M\rrbracket\rho &=& \lambda h . \lambda \varphi . \mathcal{B}\llbracket{}M\rrbracket\varphi\left(\lambda r . \begin{cases}
                                                                                                                                               h~ e&\text{if }r = \mathbf{inl}~ e\\
                                                                                                                                               \mathcal{U}\llbracket{}M\rrbracket r&\text{otherwise}
                                                                                                                                              \end{cases}\right)\\
\llbracket{}\mathbf{rdmem}_M\rrbracket\rho &=& \lambda \alpha . \lambda \mu . \mathcal{U}\llbracket{}M\rrbracket (\mu\alpha,\mu)\\
\llbracket{}\mathbf{wrmem}_M\rrbracket\rho &=& \lambda \alpha . \lambda \omega . \lambda \mu . \mathcal{U}\llbracket{}M\rrbracket ((),\mu[\alpha \mapsto \omega])
\end{eqnarray*}

\section{Translation Scheme to RWNF}

\end{document}
